%%%%%%%%%%%%%%%%%%%%%%%%%%%%%%%%%%%%%%%%%%%%%%%%%%%%%%%%%%%
% EPFL report package, main thesis file
% Goal: provide formatting for theses and project reports
% Author: Mathias Payer <mathias.payer@epfl.ch>
%
% To avoid any implication, this template is released into the
% public domain / CC0, whatever is most convenient for the author
% using this template.
%
%%%%%%%%%%%%%%%%%%%%%%%%%%%%%%%%%%%%%%%%%%%%%%%%%%%%%%%%%%%
\documentclass[a4paper,11pt,oneside]{report}
% Options: MScThesis, BScThesis, MScProject, BScProject
\usepackage[MScProject,lablogo]{EPFLreport}
\usepackage{xspace}
\usepackage{xcolor}
\usepackage{listofitems}
\usepackage{subcaption}

\newif\ifreview
\reviewtrue
% To remove the reviews seet \reviewfalse
%\reviewfalse

\definecolor{amber}{rgb}{1.0, 0.75, 0.0}
\newcounter{ReviewerID}
\readlist*\annotationcolors{blue, red, orange, green, purple, amber, brown, olive}
\newcommand{\newreviewer}[2]{%
	\ifnum \theReviewerID=\annotationcolorslen
	\setcounter{ReviewerID}{0} % cycle through colors
	\fi
	\stepcounter{ReviewerID}%
	\expandafter\edef\csname bootstrap#1\endcsname{%
		\expandafter\def\csname #1\endcsname####1{%
			\ifreview%
			 {\noexpand\color{\annotationcolors[\theReviewerID]} {\noexpand\bf{\noexpand\fbox{#2}} {\noexpand\it ####1} }}
			\else%
			 {}% disable annotations
			\fi%
		}%
	}%
	\csname bootstrap#1\endcsname%
}

\newreviewer{philipp}{philipp}

\title{House of Scudo}
\author{Elias Valentin Boschung}
\supervisor{Mao Philipp Yuxiang}
\adviser{Prof. Dr. sc. ETH Mathias Payer}
%\coadviser{Second Adviser}
%\expert{The External Reviewer}

\newcommand{\sysname}{House of Scudo\xspace}

\begin{document}
\maketitle{}
\dedication{
  \begin{raggedleft}
    If debugging is the process of removing software bugs,\\
    then programming must be the process of putting them in.\\
    --- Edsger Dijkstra\\
  \end{raggedleft}
  \vspace{4cm}
  \begin{center}
    Dedicated to my study companion since Covid-19, my plush bear.
  \end{center}
}
\makededication{}
\acknowledgments{
I thank first and foremost my family for their unconditional
support throughout all of my studies until this point. I also thank my
supervisor Philipp for all the tips and help in writing this project.
Thanks to my friends, those who inspired and encouraged me on the path of
cybersecurity, as well as those who helped me relax when I was stressed.
Thanks also to the HexHive lab for the opportunity to do this Master Semester
Project, and also for providing the template for this report, which is
available at \url{https://github.com/hexhive/thesis_template}.}
\makeacks{}

\begin{abstract}

  As a followup to the Bachelor Project for creating tooling for the Scudo
  allocator, this project focuses on using the tool and the acquired knowledge
  of Scudo to try to find exploits of heap errors, similar to existing
  exploitation techniques for the standard libc allocator.

  Using the fact that the checksum used to secure the header of a chunk is
  breakable with a single header leak, we introduce two exploits that
  use a free operation where we can control preceding bytes to get control over
  the next chunk allocation locations.

  One of these two exploits has already been fixed in the latest Scudo version,
  while we propose a mitigation for the second one.
  
\end{abstract}

\maketoc{}

%%%%%%%%%%%%%%%%%%%%%%
\chapter{Introduction}
%%%%%%%%%%%%%%%%%%%%%%



%%%%%%%%%%%%%%%%%%%%%%%%%%%%%%%%%
\chapter{Scudo Security Measures}
%%%%%%%%%%%%%%%%%%%%%%%%%%%%%%%%%

Talk about chunk header integrity checks, checksum, class id, cache

%%%%%%%%%%%%%%%%%%%%%%%%%%%%%
\chapter{Breaking the Cookie}
%%%%%%%%%%%%%%%%%%%%%%%%%%%%%

To check the integrity of chunk headers, Scudo uses a CRC32 checksum
whenever hardware CRC32 is available, otherwise it instead uses a software
BSD checksum. Both of these checksum algorithms are similar in that they
aren't cryptographically secure. The cookie used to calculate the checksum
is 32 bits long and used as the starting value for the CRC32 (or BSD) checksum.
However the actual checksum stored in a chunk header is only 16 bits long, so
in order to shorten the checksum the XOR of the lower 16 bits with the higher
16 bits is taken.

Due to the checksum only being 16 bits, instead of the full 32 bits from the
cookie, it is possible to find a cookie collision and produce the same checksums
the original cookie would have produced. Indeed it is enough to get a single
heap leak of a complete chunk header together with its address. To accomplish
that we can simply bruteforce the cookie value until we find the same checksum
that we leaked. Using a C program this takes around 1-2 seconds, however due to
the poor performance of loops in python, the bruteforce does not natively work
in python. Therefore we created a small python library written in native C code,
which can be used to calculate checksums and bruteforce a cookie given the header
leak data from within python.

%%%%%%%%%%%%%%%%%%
\chapter{Exploits}
%%%%%%%%%%%%%%%%%%

\section{Change Class ID}



\section{Safe Unlink Secondary}


\subsection{Summary}

- Header leak with corresponding address
- Scudo library base leak
- Three consecutive frees on the same address
  - Control over the $0x10$ bytes in front of the address before the first two free's
  - Control over the $0x40$ bytes in front of the address before the third free

Result:
Next allocation will be a chunk in the perclass structure itself (thread-local free list), allowing control of future allocations


\subsection{Requirements}

The safe unlink exploit has a certain number of requirements that the target binary needs to fulfill.

First of all we need a header leak of any chunk from the heap as well as it's address, in order to bruteforce the cookie. Afterwards we can forge any header checksum from the cookie we calculated from this leak.

Second we need a free for which we can control a certain number of bytes before the free, more specifically we need to control $0x40$ bytes in front of the address of the free. This is the size of the secondary header plus the size of the padded primary header.

Third we need two interesting places in memory (at addresses `add1` and `add2` resepectively) that have the address of the free address from the previous step stored. We will store $add2$ at $add1+0x8$ and $add1$ at $add2$. Therefore `add1` and `add2` should be in some interesting location allowing elevation of our access.  
The easiest way to get this is by having two more free's of the chunk in the previous step and being able to control the $0x10$ bytes in front of the address to modify the header. With this method we however also need a leak of the scudo lib base, in order to get the location of the thread-local free list (PerClass list).

\subsection{Explanation}

As with most scudo exploits, the first step is to get the cookie, which can be done from any single header leak, as explained in the dedicated section.

Then we need to have two locations in memory that point to the chunk we will tamper with in the next section. The way we achieve this is by having two frees on that same chunk, leaving it's address twice in the perclass structure next to each other. The perclass structure constitutes the first level, thread specific free list of chunks. It is handled as a simple array of chunk addresses, and therefore freeing the same chunk twice leaves two pointers to that chunk right next to each other in the perclass structure. This setup is especially interesting since getting the allocation of a chunk in the perclass structure could allow us to control all following allocations and to allocate chunks at arbitrary addresses. The only obstacle is that we need to modify the chunk header before the second free to mark the chunk as allocated again.

Once we have setup the two addresses in the perclass array, we need to know it's address still. To calculate the address of the perclass array (s) we need to have a leak of some scudo address, like the base address where scudo is loaded or the address of the malloc function. We can calculate the offset needed from there based on the scudo version, and then advance by $0x100$ times the value of the class id where we free'd the chunks. Then we just need to guess the number of chunks present in the perclass structure of that class id.

Finally we need to prepare the header of the final free that will trigger the safe unlink. For that we need to configure a fake secondary header in front of the primary header, as well as modifying the latter to set it's class id to 0. We set the prev ($chunk_addr - 0x40$) to the first of our locations $- 0x8$ and the next ($chunk_addr - 0x38$) to the second of our locations. With the perclass setup, we can set both of them to the address where the first of the two addresses is stored.
Next follow the CommitBase, CommitSize and MapBase, MapSize. We can set them to the values we want, we can set them to whatever since we don't actually use them. They have a size of $0x8$ bytes each.
Finally we just need to set the primary header with the class id set to 0 and the checksum recalculated.

When the free of that chunk then happens, it is handed to the secondary allocator since we set the class id to 0. The secondary allocator tries to remove the chunk from the linked list of in use chunks, even when our fake chunk never was in it. So it tries to set the Next->Prev pointer to Prev and the Prev->Next pointer to Next, which leads it to write the addresses of our locations in the perclass structure to those same addresses, which will allow us to allocate a chunk in the perclass structure itself.



\section{Exploit CommitBase}

\subsection{Summary}

- Header leak with corresponding address
- At least one secondary chunk allocated
- Free on any address where we control `0x40` bytes in front
- Allocation of secondary chunk of known (approximate) size

Result:
We can choose an arbitrary location for the secondary chunk allocated

\subsection{Requirements}

The forge secondary exploit has a certain number of requirements that the target binary needs to fulfill.

First of all we need a header leak of any chunk from the heap as well as it's address, in order to bruteforce the cookie. Afterwards we can forge any header checksum from the cookie we calculated from this leak.

Second we need a free for which we can control a certain number of bytes before the free, more specifically we need to control `0x40` bytes in front of the address of the free. This is the size of the secondary header plus the size of the padded primary header.

Third we need there to be at least one secondary chunk allocated before the free, and we need there to be another secondary allocation afterwards, of which we know the approximate size. This is the allocation where we will get our arbitrary access.




%%%%%%%%%%%%%%%%%%%%
\chapter{Mitigation}
%%%%%%%%%%%%%%%%%%%%



\section{Benchmark}

In order to benchmark the changes of the mitigation, we used some custom benchmarks
to test for impact with very targeted programs, as well as a benchmarking suite
for malloc implementations, mimalloc-bench, which is usually used to compare many
different allocators between each other.~\cite{mimalloc-bench} We slightly modified
mimalloc-bench to include our patched scudo version, and ran the benchmarks between
the latest scudo version and our patched version.


%%%%%%%%%%%%%%%%%%%%
\chapter{Conclusion}
%%%%%%%%%%%%%%%%%%%%



\cleardoublepage{}
\phantomsection{}
\addcontentsline{toc}{chapter}{Bibliography}
\printbibliography{}

\end{document}
